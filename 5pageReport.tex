%Overarching document template
\documentclass[11pt]{article}

%%%%%%%%%%%%%%%%%%%%%%%%
% Package Inclusion                                        
%%%%%%%%%%%%%%%%%%%%%%%%

\usepackage{amsmath} % Required for many math properties. AMS stands for American Mathematical Society.
\usepackage{amssymb} %Required for using many math symbols.
\usepackage{geometry} % Required for adjusting page dimensions.
\usepackage[utf8]{inputenc} % Required for inputting international characters.
\usepackage[none]{hyphenat} % Removes automatic hyphenation (a personal pet peeve of Sam).
\usepackage{hyperref} % Required to include URLs
\usepackage{xcolor} % Required to turn URLs the correct colors.
\usepackage{listings} % Required to display code.
\usepackage{bm} % Required to make vectors bold.
\usepackage{titlesec}
\usepackage{graphicx}
\usepackage{minted}
%%%%%%%%%%%%%%%%%%%%%%%%
% Page settings                                              
%%%%%%%%%%%%%%%%%%%%%%%%

% Set up page dimensions.
\geometry{
    paper=letterpaper, % 8.5 x 11 paper.
	top=1in, % Top margin.
	bottom=1in, % Bottom margin.
	left=1in, % Left margin.
	right=1in, % Right margin.
}
 % Set up code display settings
\lstset{
frame = single, % Box the code
numbers=left % Show line numbers
}
% Set up line and paragraph spacing
\setlength{\parindent}{0em} % Sets paragraph indentation
\setlength{\parskip}{1em} % Sets paragraph spacing
\linespread{1.0} % Sets line spacing

% Document attributes
\title{CS 5500 Project Report: MPI Password Cracker}
\author{Andrew Knoblach \\ AMKnoblach@gmail.com \and  Carter McGee \\ cartermcgee7@gmail.com}
\date{\today}

% Section formatting
\titleformat{\section}{\normalfont\Large\bfseries}{}{0em}{}

%%%%%%%%%%%%%%%%%%%%%%%%%%%%%%%%%%%%
% Begin the content which will be displayed in the output PDF
% Only things in between \begin{document} and \end{document} actually show up on your paper!
%%%%%%%%%%%%%%%%%%%%%%%%%%%%%%%%%%%%
\begin{document}
\maketitle

\section{Project Description}

The goal of this project was to create a simple brute force password cracker. The reason we chose this is because password cracking shows up on the list of Embarrassingly Parallel problems. The idea was simple, take a hashed password and try every single combination until you find a string that when hashed matches your original hash.

For this project we choose to use an MD5 hash as it is a common hash that isn't too computationally expensive. Although MD5 is no longer considered a secure hash (and hasn't been for a while) we figured it was good for our proof of concept.



\section{Overview of Code}

Our password cracker program is broken up into several bits.  The first bit is concerned with gathering information and sending it out to all the processes. In this part we gather:
\begin{itemize}
	\item The length of the password
	\item If the password space contains lowercase letters
	\item If the password space contains uppercase letters
	\item If the password space contains numbers
	\item The hash to be broken
\end{itemize}

All of these settings are gathered by rank0 then using \verb|MPI_Bcast| the settings are sent to all other processes and \verb|MPI_Send| is used to send the hash to all other processes.

After this section the password spaces are created according to what was dictated in the settings. The password spaces use the ASCII values for each character and then are appended to a password space vector. For this project we only included \verb|a-zA-Z0-9| the math behind this decision is covered in the next section. 

Next the actual password cracking begins. Using a recursive function each possible string of length $n$ is generated within the given password space. Once a string is generated it is then hashed. The hash is then compared to the original hash provided by the user. If the hash matches the program reports the string that gave a match as well as how long the crack function was running to find this string. 

Lastly the program will inform all other processes that the correct hash has been found and that it is time to terminate. 

\section{Testing}
First, we would like to explain why we set our password space as we did. With the characters \verb|a-zA-Z0-9|the password space maxes out at 62 possible characters for each space in the password. This makes the number of possible passwords $62^n$ where $n$ is the length of the password. To give an idea of how many passwords that contains here is a graph of $62^n$. 

\begin{center}
	\includegraphics[scale=0.5]{ngraph.png}
\end{center}

Observe that there is a large spike around $n=6$ due to this spike we did not test many passwords over length of 6 as passwords of length 6 already bordered on taking an hour to crack. 

We split testing up into 4 different categories:
\begin{itemize}
	\item Only A-Z, Only 0-9, Only a-z
	\item Lower Case with numbers
	\item Upper and Lower case
	\item Uppercase, Lower, and numbers
\end{itemize}

We then tested each of categories with passwords of upto $n=6$ and with 2,4, \& 8 cores. The results were as follows.

Note: Some times listed in milliseconds as processes finished faster than a second Or in some cases measuring in milliseconds gave a better indication of time difference 

\begin{center}
 \begin{tabular}{||c c c c c||} 
 \hline
 
 \multicolumn{5}{|c|}{Only A-Z, Only a-z, Only 0-9} \\
 \hline
 \hline
 Password & Length & Time(2 Processes) & Time(4 Processes) & Time(8 Processes) \\ [0.5ex] 
 \hline\hline
 a & 1 & 0s & 0s & 0s\\ 
 \hline
 yz & 2 & 0s & 0s & 0s\\
 \hline
 qsc & 3 & 5ms & 6ms & 5ms\\
 \hline
 test & 4 & 250ms & 43ms & 91ms\\
 \hline
 qwert & 5 & 3625ms & 5056ms & 3762ms\\ 
 \hline
 asdfgh & 6 & 17s & 19s & 36s\\
 \hline 
  5 & 1 & 0s & 0s & 0s\\
  \hline 
  43 & 2 & 0s & 0s & 0s \\
  \hline
  733 & 3 & 0s & 0s & 0s\\
  \hline
  1252 & 4 & 2ms & 2ms & 1ms\\
  \hline
  99054 & 5 & 101ms & 80ms & 110ms\\
  \hline 
  645367 & 6 & 300ms & 102ms & 199ms\\
  \hline
  U & 1 & 0s & 0s & 0s\\
  \hline
  YS & 2 & 0s & 0s & 0s\\
  \hline
  LIX & 3 & 17ms & 7ms & 7ms\\
  \hline
  LSOX & 4 & 424ms & 204ms & 210ms\\
  \hline
  KIXES & 5 & 9657ms & 4383ms & 2722ms\\
  \hline 
  SEDVDE & 6 & 126s & 3873ms &  6720ms\\
 [1ex] 
 \hline
\end{tabular}
\end{center}


\begin{center}
 \begin{tabular}{||c c c c c||} 
 \hline
 
 \multicolumn{5}{|c|}{Lower Case \& Numbers} \\
 \hline
 \hline
 Password & Length & Time(2 Processes) & Time(4 Processes) & Time(8 Processes) \\ [0.5ex] 
 \hline\hline
 8 & 1 & 0s & 0s & 0s\\ 
 \hline
 3c & 2 & 0s & 0s & 0s\\
 \hline
 10b & 3 & 25ms & 1ms & 21ms\\
 \hline
 8ggw & 4 & 1470ms & 679ms & 917ms\\
 \hline
 98ffe & 5 & 61s & 33s & 38s\\ 
 \hline
 bnd3d0 & 6 & 170s & 226s & 300s\\
 [1ex] 
 \hline
\end{tabular}
\end{center}

\begin{center}
 \begin{tabular}{||c c c c c||} 
 \hline
 
 \multicolumn{5}{|c|}{Upper and Lower Case} \\
 \hline
 \hline
 Password & Length & Time(2 Processes) & Time(4 Processes) & Time(8 Processes) \\ [0.5ex] 
 \hline\hline
 Z & 1 & 0s & 0s & 0s\\ 
 \hline
 wC & 2 & 2ms & 0s & 0s\\
 \hline
 QoD & 3 & 87ms & 18ms & 3ms\\
 \hline
 aFEM & 4 & 173ms & 290ms & 303ms\\
 \hline
 QQZZo & 5 & 264s & 77s & 27s\\ 
 \hline
 dadEES & 6 & 2360s & 3595s & \\
 [1ex] 
 \hline
\end{tabular}
\end{center}

\begin{center}
 \begin{tabular}{||c c c c c||} 
 \hline
 
 \multicolumn{5}{|c|}{Upper Case, Lower Case, \& Numbers} \\
 \hline
 \hline
 Password & Length & Time(2 Processes) & Time(4 Processes) & Time(8 Processes) \\ [0.5ex] 
 \hline\hline
 Z & 1 & 0s & 0s & 0s\\ 
 \hline
 3c & 2 & 0ms & 0s & 0s\\
 \hline
 1Gb & 3 & 175ms & 66ms & 84ms\\
 \hline
 F3sZ & 4 & 422ms & 1357ms & 4115ms\\
 \hline
 d5pZ1 & 5 & 177s & 159s & 258s\\ 
 [1ex] 
 \hline
\end{tabular}
\end{center}

\section{Conclusion}

In conclusion, our parallelized password cracker does actually work. The password cracker can crack 5 character long passwords in about a few minutes. With more time to test we imagine it might be able to handle 7-8 character long passwords if on larger machines or given a greater period of time to spend cracking. In most cases during our testing adding more processors did reduce the time needed to crack the password, although not always. The likely reason that it did not always reduce the time needed to crack the password is that we were running this program on our work stations and the presence of other processes may have taken some of the compute time away from the password cracker, though this could be tested further. All in all, we think the project is quite a success.Though, if we were to continue with it, there are some further improvements and testing that we would do.


\section{Appendix-A Code}
\begin{minted}{C++}
#include <mpi.h>
#include <stdio.h>
#include <iostream>
#include <stdlib.h>
#include <string>
#include <cstring>
#include <vector>
#include "md5.h"
#include <chrono>

#define MCW MPI_COMM_WORLD
#define TERMINATE 8

MPI_Status myStatus;

std::string hash;
int startIndex;
int endIndex;
int passwordLength = 6;
int size, rank, data;
int myFlag;
auto start = std::chrono::high_resolution_clock::now();

void generate(char* arr, int i, std::string s, int len) {
    // base case
    if (i == 0){ // when len has been reached
        // check if the password was found
        if(hash.compare(md5(s)) == 0){
            auto stop = std::chrono::high_resolution_clock::now();
            auto durationSeconds = std::chrono::duration_cast<std::chrono::seconds>(stop -start);
            auto durationMs = std::chrono::duration_cast<std::chrono::milliseconds>(stop-start);
            std::cout << "The password is: " << s << std::endl << std::endl;
            std::cout << "It took " << durationSeconds.count() << " seconds to find" << std::endl;
            std::cout << "It took " << durationMs.count() << " miliseconds to find" << std::endl;
            // tell all the other processes to terminate
            for(int i = 0; i < size; i++){
                if(i != rank){
                    MPI_Send(&data, 1, MPI_INT, i, TERMINATE, MCW);
                }
            }
            MPI_Finalize();
            exit(0);

        } else {
            // check if a process found the password
            MPI_Iprobe(MPI_ANY_SOURCE, TERMINATE, MCW, &myFlag, &myStatus);
            if(myFlag){
                MPI_Finalize();
                exit(0);
            }
        }
        return;
    }

    // iterate through the array
    for (int j = 0; j < len; j++) {

        // Create new string with next character
        // Call generate again until string has
        // reached its len
        std::string appended = s + arr[j];
        generate(arr, i - 1, appended, len);
    }

    return;
}

// function to generate all possible passwords
void crack(char* arr, int len) {
    for(int i = startIndex; i <= endIndex; i++){
        std::string startString = "";
        startString += static_cast<char>(arr[i]);
        generate(arr, passwordLength - 1, startString, len);
    }
}


int main(int argc, char **argv) {

	bool includeLower = false;
	bool includeUpper = false;
	bool includeNumbers = false;

	//MPI INIT Stuff
	MPI_Init(&argc, &argv);
	MPI_Comm_size(MCW, &size);
	MPI_Comm_rank(MCW, &rank);
    int hashLength = -1;
    std::string yes = "Y";

	//Get settings in rank 0 then distribute settings to other ranks
	if(rank == 0){
        //std::cout << "md5 hash: " << md5("test") << std::endl;
		//Settings 
		// -Known password length
		// -Provide password hash
		// For now, assume the password has upper case, lower case, and numbers
		bool havePassLen = false;
		while(!havePassLen){
            std::cout<< "\nHow long is the password? " ;
            std::cin >> passwordLength;

            if(passwordLength < 1){
                std::cout << "Please input a password length of at least length 1." << std::endl;
            } else{
                havePassLen = true;
            }
		}

		bool validPWS = false;
		while(!validPWS){
            std::string lower = "";
            std::cout << "Does the password contain lower case characters? (Y/n) ";
            std::cin >> lower;
            if(yes.compare(lower) == 0){
                includeLower = true;
            }

            std::string upper = "";
            std::cout << "Does the password contain upper case characters? (Y/n) ";
            std::cin >> upper;
            if(yes.compare(upper) == 0){
                includeUpper = true;
            }

            std::string numbers = "";
            std::cout << "Does the password contain numbers? (Y/n) ";
            std::cin >> numbers;
            if(yes.compare(numbers) == 0){
                includeNumbers = true;
            }

            if(!includeLower && !includeUpper && !includeNumbers){
                std::cout << "Please enter a valid password space." << std::endl;
            } else {
                validPWS = true;
            }
		}

        //flush buffer
        std::string s;
        std::getline(std::cin, s);

		std::cout << "What is the password hash? ";
		std::getline(std::cin, hash);
		hashLength = hash.size();
	}

	// make sure every process knows the password space
    MPI_Bcast(&includeLower, 1, MPI_C_BOOL, 0, MCW);
    MPI_Bcast(&includeUpper, 1, MPI_C_BOOL, 0, MCW);
    MPI_Bcast(&includeNumbers, 1, MPI_C_BOOL, 0, MCW);

    // make sure every process has the length of the hash string
    MPI_Bcast(&hashLength, 1, MPI_INT, 0, MCW);

    // send/recv the hash string as a char array
    if(rank == 0){
        for(int i = 0; i < size; i++){
            MPI_Send(hash.c_str(), hash.size(), MPI_CHAR, i, 0, MCW);
        }
    }else {
        char hashCharArr[hashLength];
        MPI_Recv(hashCharArr, hashLength, MPI_CHAR, 0, 0, MCW, MPI_STATUS_IGNORE);
        hash = hashCharArr;
    }

    // broadcast the password length
    MPI_Bcast(&passwordLength, 1, MPI_INT, 0, MCW);

    //Calculate password space
    std::vector<int> passwordSpace;
    if(includeLower){
        for(int i = 97; i < 123; i++){ passwordSpace.push_back(i);} // ascii values a-z
    }
    if(includeUpper){
        for(int i = 65; i < 91; i++){ passwordSpace.push_back(i);} // ascii values A-Z
    }
    if(includeNumbers){
        for(int i = 48; i < 58; i++){ passwordSpace.push_back(i);} // ascii values 0-9
    }

    startIndex = (passwordSpace.size() / size) * rank;
    if(rank == (size - 1)){
        endIndex = passwordSpace.size() - 1;
    } else {
        endIndex = (startIndex + (passwordSpace.size() / size));
    }

    char pws[passwordSpace.size()];
    for(int i = 0; i < passwordSpace.size(); i++){
        pws[i] = static_cast<char>(passwordSpace[i]);
    }

    // find the password
    ::start = std::chrono::high_resolution_clock::now();
    crack(pws, passwordSpace.size());

    MPI_Finalize();
	return 0;
}

\end{minted}

\end{document}